\documentclass[12pt,a4paper]{article}
% AUTHOR: Rafael Belchior
% Thanks to Prof. RUI SANTOS CRUZ for providing the template
%
\usepackage{helvet} 
\renewcommand{\familydefault}{\sfdefault}
\usepackage{a4wide}
\usepackage{ucs}
\usepackage[utf8x]{inputenc}
\usepackage{amsmath}

%%%%%%%%%%%%%%%%%%%%%%%%%%%%%%%%%%%%%%%%%%%%%%%%%%%%%%%%%%%%%%%%%%%%%%%%%%%%%%%%%%%
% SELECT ONE OF THE FOLLOWING PACKAGES FOR THE LANGUAGE 
\usepackage[english]{babel}
% \usepackage[portuges]{babel}
%%%%%%%%%%%%%%%%%%%%%%%%%%%%%%%%%%%%%%%%%%%%%%%%%%%%%%%%%%%%%%%%%%%%%%%%%%%%%%%%%%
\usepackage{subfig}
\usepackage{graphicx}
\usepackage{hyperref}
\usepackage{cite}
\usepackage[absolute]{textpos}
\usepackage{tabularx} 
\usepackage{tabulary}                 
\usepackage{fancyhdr}
\usepackage[table]{xcolor}
\pagestyle{fancy}
\headsep=50pt
\setlength{\headheight}{50pt}
\usepackage{listings}
\usepackage{minted}
\definecolor{LightGray}{rgb}{0.95, 0.95, 0.95}
\definecolor{darkblue}{rgb}{0.0,0.0,0.6}
\definecolor{editorOcher}{rgb}{1, 0.5, 0}

% Clever Referencing of document parts
\usepackage{cleveref}

\lstdefinestyle{commandline} {%
language={[WinXP]command.com},
breaklines=true,
%aboveskip=\baselineskip,
belowskip=\baselineskip,
showstringspaces=false,
backgroundcolor=\color{LightGray},
basicstyle=\small\color{black}\ttfamily,
showstringspaces=false,
keywordstyle=\color{cyan}\bfseries,
stringstyle=\color{cyan}\ttfamily,
commentstyle=\color{green}\itshape,
moredelim=[s][\color{blue}\bfseries]{C:}{\>}
}

\lstdefinestyle{Bash} {%
language=bash,
breaklines=true,
belowskip=\baselineskip,
backgroundcolor=\color{LightGray},
showstringspaces=false,
keywordstyle=\color{black}\bfseries,
basicstyle=\small\color{black}\ttfamily,
stringstyle=\color{editorOcher}\ttfamily,
commentstyle=\color{cyan}\itshape,
otherkeywords={xcode-select, mkdir,rm},
moredelim=[s][\color{red}]{~$},
literate={~} {$\sim$}{1}
}
%%%%%%%%%%%%%%%%%%%%%%%%%%%%%%%%%%%%%%%%%%%%%%%%%%%%%%%%%%%%%%%%%%%%%%%%%%%%%%%%%%%
% PLEASE FILL THE ADEQUATE DATA IN THE TABLE REPLACING
% THE VALUES EXEMPLIFIED
\lhead{}
{\renewcommand{\arraystretch}{1.1}
\fancyhead[C]{\begin{tabularx}{1.0\textwidth}{|l|X|l|l|}
\hline 
% In the following line change Course Name: PPIII, PPB
\textbf{EB 20/21} & \textbf{Enterprise Blockchain Technologies} & \textbf{Number:}  &  2\\
\hline
% In the following line insert your Name and IST ID
\multicolumn{2}{|l|}{Module I - Introduction} & \textbf{Issue Date:}  &  21 Sept 2020 \\ 
\hline
% In the following line insert the Activity CODE and Title (abridged)
%\textbf{WP n.} (99) & (Subject) & \textbf{Group:} & (99) \\
\multicolumn{2}{|l|}{Background: Cryptography \& Security} & \textbf{Due Date:} &  28 Sept 2020\\ 
\hline
\end{tabularx}}
\rhead{}

%%%%%%%%%%%%%%%%%%%%%%%%%%%%%%%%%%%%%%%%%%%%%%%%%%%%%%%%%%%%%%%%%%%%%%%%%%%%%%%%%%%
% DO NOT CHANGE THIS BLOCK
\begin{document}
\textblockorigin{-34pt}{-12pt}
\begin{textblock*}{10cm}(2cm,1cm)
\includegraphics[width=6cm]{hyperledger.png}
\end{textblock*}
%%%%%%%%%%%%%%%%%%%%%%%%%%%%%%%%%%%%%%%%%%%%%%%%%%%%%%%%%%%%%%%%%%%%%%%%%%%%%%%%%%%,sdist2017

\section*{Instructors Guide}
This document provides a proposal of solution for LAB\#02, which concerns the an introduction to crypography and security. In particular, the RSA algorithm is introduced. Most of the solutions can be found at \cite{conrad2016,rogaway2004,md2020}. Note: You can also experiment with Hyperledger Ursa, a crypto library \footnote{https://github.com/hyperledger/ursa}, which supports symmetric encryption and digital signatures. 



%%%%%%%%%%%%%%%%%%%%%%%%%%%%%%%%%%%%%%%%%%%%%%%%%%%%%%%%%%%%%%%%%%%%%%%%%%%%%%%%%%%
% YOUR TEXT STARTS HERE
%%%%%%%%%%%%%%%%%%%%%%%%%%%%%%%%%%%%%%%%%%%%%%%%%%%%%%%%%%%%%%%%%%%%%%%%%%%%%%%%%%%   

%%%%%%%%%%%%%%%%%%%%%%%%%%%%%%%%%%%%%%%%%%%%%%%%%%%%%%%%%%%%%%%%%%%%%%%%%%%%%%%%%%

\section{RSA}
\label{sec:rsa}

\subsubsection*{Exercise 1: How many combinations can the MD5 and SHA256-3 algorithms generate? What is the likelihood of a hash collision with MD5? And with SHA256-3? Is SHA256-3 safer than MD5?}

MD5 is a 128-bit cryptographic hash function, which means that digests are distributed over a 128-bit space. MD5 can output $2^{128}$ digests. Two files should have a $\frac{1}{2^{128}}$ change of collision. However, due to the birthday paradox, that probability would be $\frac{1}{2^{64}}$.

SHA256-3 is a 256-bit cryptographic hash function, which means that digests are distributed over a 256-bit space. SHA256-3 can output $2^{256}$ different digests. 
Two files should have a $\frac{1}{2^{256}}$ change of collision. However, due to the birthday paradox, that probability would be $\frac{1}{2^{128}}$.

MD5 is considered to be vulnerable to collision attacks \cite{wang2005}.


\subsubsection*{Exercise 2: Refer to Figures 3 and 4. Is this approach of signing and validating a document secure?}

This approach is generally considered secure, given that Alice's public key is authentic (to prevent man-in-the-middle attacks), and that the hash function that creates the digest is secure. The key distribution should be done in a secure way. See public-key infrastructure, public key certificates, and the X.509 norm. 

\subsubsection*{Exercise 4: Calculate the RSA keys associated with primes $p = 7$ and $q = 29$}
Public key is $K_u = (N,e)$ and the private key, $K_r = (N,d)$.

$N = p.q = 7.29 = 203$
$Z = (p - 1).(q - 1) = 6.28 = 168$
$e$ is chosen such that the greatest common divisor between e and 168 is be one. 
$e = 5, 11, 13, 17, 19, ... 157, 163, 167$.
$d = 101$, because 101 is the only number such that $e.d \mod (p - 1).(q - 1) = 1$. Restriction: $0 < d < (p - 1).(q - 1)$.
The public key $K_u = (N,e)$ is $(203, 5)$ and the private key, $K_r = (N,d)$ is $(203,101)$.

\subsubsection*{Exercise 5: Given that the public key $K_u = (N,e) = (143, 7)$, encrypt the following message: ``P''}
P is 80 in ASCII encoding.
$C = M^e \mod n = 80^7 mod 143 = 141$.

\subsubsection*{Exercise 6: Decrypt the criptogram 80, given that $p = 3$ and $q = 31$ }
$N = p.q = 3.31 = 93$.
$d = 43$.
The private key is, therefore, $K_r = (N,d) = (93,43)$.
$M = C^d \mod n = 80^43 mod 93 = 87$.

%\subsubsection*{Exercise 7: Adapt the code to support text encrpytion in the ASCII format, and separation into different blocks}


%%%%%%%%%%%%%%%%%%%%%%%%%%%%%%%%%%%%%%%%%%%%%%%%%%%%%%%%%%%%%%%%%%%%%%%%%%%%%%%%%%%
% YOUR TEXT ENDS HERE
%%%%%%%%%%%%%%%%%%%%%%%%%%%%%%%%%%%%%%%%%%%%%%%%%%%%%%%%%%%%%%%%%%%%%%%%%%%%%%%%%%% 
\bibliographystyle{IEEEtran}
\bibliography{lab.bib}

\end{document}                             % The required last line